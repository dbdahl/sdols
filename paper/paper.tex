\documentclass[sts]{imsart}

\usepackage{amsthm,amsmath,amssymb,natbib}
\RequirePackage[colorlinks,citecolor=blue,urlcolor=blue]{hyperref}

\startlocaldefs
\endlocaldefs

\begin{document}

\begin{frontmatter}

\title{Summarizing Distributions of Latent Structure}
\runtitle{Summarizing Distributions of Latent Structure}

\author{\fnms{David B.} \snm{Dahl}\corref{}\ead[label=e1]{dahl@stat.byu.edu}}
\address{\printead{e1}}
\vspace{-1ex}
\affiliation{Brigham Young University}

\author{\fnms{Peter} \snm{M\"uller}\ead[label=e2]{pmueller@math.utexas.edu}}
\address{\printead{e2}}
\affiliation{University of Texas - Austin}

\runauthor{Dahl, M\"uller}

\begin{abstract} In a typical Bayesian analysis, consider effort may be placed
on ``fitting the model'' (e.g., obtaining samples from the posterior
distribution) but this is only half of the inference problem.  Typically
meaningful inference also requires summarizing the posterior distribution of
the parameters of interest.  Posterior summaries can be especially important in
communicating the results and conclusions for a Bayesian analysis to a diverse
audience.  If the parameters of interest live in $\mathbb{R}^n$, common
posterior summaries are means, medians, and modes.  Summarizing posterior
distributions of parameters with complicated structure is a more difficult
problem.  For example, the ``average'' network in the posterior distribution on
a network is not easily defined.  This paper reviews methods for summarizing
distributions of latent structure and provides a general framework which we
apply specifically to distributions on variable selection indicators,
partitions, feature allocations, and networks.  We illustrate our approach in a
variety of models for both simulated and real datasets.  \end{abstract}

%\begin{keyword}[class=MSC]
%\kwd[Primary ]{}
%\kwd{}
%\kwd[; secondary ]{}
%\end{keyword}

\begin{keyword}
\kwd{decision theory}
\kwd{least-squares clustering}
\kwd{loss functions}
\kwd{MAP estimation}
\kwd{posterior summaries}
\end{keyword}

\end{frontmatter}

Idea about the exemplar observation.

Variable selection from Bamboo project.

\end{document}

